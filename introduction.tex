\documentclass[a4paper]{article}

%% Language and font encodings
\usepackage[english]{babel}
\usepackage[utf8x]{inputenc}
% \usepackage[T1]{fontenc}
\hyphenation{trans-for-ma-tion}

%% Sets page size and margins
\usepackage[a4paper,top=3cm,bottom=2cm,left=3cm,right=3cm,marginparwidth=1.75cm]{geometry}

%% Useful packages
\usepackage{amsmath}
\usepackage{graphicx}
\usepackage[colorlinks=true, allcolors=blue]{hyperref}
\usepackage{apacite} % APA style citations
\AtBeginDocument{\urlstyle{APACsame}}  % Links in APA citytions same formatting
\usepackage{tabulary}
\usepackage{natbib} % natbib citations: \citep{} and \citet{} for in-text
\usepackage{booktabs}
% If you use \tableofcontents, this adjusts the name
%\addto\captionsenglish{\renewcommand*\contentsname{Table of Contents}}



\begin{document}

\pagenumbering{gobble}

Africa, a vast continent of 30.4 million km$^2$, home to 1.5 billion diverse peoples in 54 recognized states, the median of which is 19 years of age today, has grown economically at an average rate of 4.1\% between 2001 and 2023 and is projected by the IMF's World Economic Outlook (WEO) to grow at rates around 4.2\% in 2024-2029 (\citet{krantz2023africamonitor}, own calculations). This performance must be compared to growth of only 2.1\% in 1980-2000. In per-capita terms, the growth rate was $\sim$0\% in 1980-2000, rose to $\sim$1.9\% in 2001-2023, and is projected to remain at that level through 2029. Thus, Africa is growing, and Africans are becoming wealthier. They have also become healthier and more educated, with an increase in life expectancy from  54.4 years in 2000 to 62 years in 2021 and an increase in expected years of schooling from 7.9 to 10.8 (\citet{krantz2023africamonitor}, UNDP data). Despite a large COVID shock reducing human development in 2020 and 2021, these long-term increases in human capital contribute to sustaining future per-capita growth. \newline 

Another reason to assume more sustained growth in Africa are the enhanced business conditions in many countries, supported by a long process of macroeconomic stabilization, i.e., a persistent decline in the volatility of real per-capita growth and inflation rates within African economies. The first paper in this dissertation \citep{krantz2023africas} documents this African Great Moderation in key macroeconomic aggregates and investigates its correlates within and across countries. The findings suggest that concurrent global moderation, improved macroeconomic policy frameworks/economic institutions, and domestic financial deepening have contributed towards macroeconomic stabilization in Africa. Classical structural change, which increased the share of the more stable service sector, and changes in agricultural technologies, inducing a decline in agricultural output volatility, also played a role. The more stable macroeconomic environment inspires greater business confidence and more secure private investing in Africa. Its preservation and fortification, despite high public debt levels, is an essential ingredient towards successful economic transformation on the continent. \newline 

Yet, with a high annual population growth of currently 2.3\% and a total population projection to reach 2.5 billion by 2050, greater economic impetus and transformation are needed to generate significant income increases that can only be supported by complex economic activities. \newline 

Such transformation has long been called for. "Africa must unite" is the mantra and title of a 1963 book by Kwame Nkrumah \citep{nkrumah1963africa}, the pan-Africanist first prime minister of Ghana, an African champion in infrastructure and industrialization, and a co-founder of the Organization of African Unity (OAU) established in the same year. Commemorating the OAU, in 2013, its successor organization, the African Union (AU), and African heads of state signed the \href{https://au.int/documents/20130613/50th-anniversary-solemn-declaration-2013}{50th Anniversary Solemn Declaration}\footnote{https://au.int/documents/20130613/50th-anniversary-solemn-declaration-2013}, re-dedicated Africa towards the attainment of the Pan African Vision of "An integrated, prosperous and peaceful Africa, driven by its own citizens, representing a dynamic force in the international arena". With \href{https://au.int/en/agenda2063/overview}{Agenda 2063}\footnote{https://au.int/en/agenda2063/overview} the AU formulated a concrete manifesto towards the attainment of this vision within 50 years, guided by 10-year implementation plans and \href{https://au.int/en/agenda2063/flagship-projects}{15 flagship projects}\footnote{https://au.int/en/agenda2063/flagship-projects} including a continental free trade area (AfCFTA), African commodities strategy, single air transport market (SAATM), integrated high-speed train network, the Grand Inga Dam, a pan-African E-network and E-university, an African economic forum, African financial institutions, and free movement of people. \newline 

As emphasized, key flagship projects revolve around advancing continental economic integration and building the necessary infrastructure. With the entry into force of the African Continental Free Trade Agreement (AfCFTA) in May 2019 and its ratification by 47 African states as of July 2024, a significant flagship project has been instigated. World Bank Estimates suggest that by 2035, AfCFTA will boost African incomes by 7\% (450B USD), total African exports by 29\%, and inner-African exports by 81\% \citep{worldbank2020afcfta}. Yet, trading under the agreement has started sluggishly. Currently, only 8 countries trade a handful of products under the Guided Trade Initiative, and many tariff reductions are outstanding. Countries' hesitation to start trading under the agreement may partly be explained by concerns about national industries and value addition.  \newline 


Thus, more evidence is needed for the optimal utilization of the agreement and the planned formulation of an African Commodities Strategy. Towards this end, a detailed and forward-looking analysis of the continent's current integration into global and regional production and trade is highly informative. The second part of this dissertation conducts such analysis utilizing the most detailed data on the global economy available to date. Notably, the EMERGING Multi-Region Input-Output (MRIO) Tables \citep{huo2022full} cover 245 economies in 135 sectors, incorporating IO tables for 23 African economies representing 84\% of African GDP, and sectoral GDP for 27 African economies. The first paper \citep{krantz2024africas}, a rather short working paper, analyzes African economic integration through trade, global and regional value chains (GVCs and RVCs) and provides the continental context for the second paper. It finds that inner-African trade has increased steadily and highlights precious stones and metals, mining (petroleum), petrochemicals, and food processing as high-potential sectors driving RVCs, with scope for further RVC expansion and deepening. The second paper \citep{krantz2024patterns} zooms in on earlier efforts of regional integration and presents a rigorous case study of global and regional integration in the East African Community (EAC). It suggests that regional integration is more beneficial than global integration but has distributional side effects, leading to a loss of competitiveness in smaller manufacturing sectors and favouring the regional manufacturing hegemon (Kenya). Hence, African countries should align domestic industrial strategies with trade liberalization and deeper engagement in trading under AfCFTA and jointly strive for industrial policy coordination. \newline    


Continental trade and competitive production and value chains in Africa will also require significant investments in infrastructure, including roads, railways, and ports for transportation, power, communications, and education for production, as well as health and public services for the general well-being of workers. However, many African governments are under financial strain to invest in infrastructure projects, with GDP-weighted average general government gross-debt levels at 64\%. The African Development Bank estimates suggest that Africa's infrastructure needs amount to \$130-170 billion a year, with a financing gap in the range of \$68-108 billion \citep{african2018africa}. International efforts such as the EU's Global Gateway Initiative have so far also failed to generate significant infrastructure investments, and Chinese investments carry high interest rates. This implies a tight public resource allocation problem both across infrastructure sectors and across space, e.g., where exactly do additional roads, power lines, schools, communications towers, etc., generate the highest economic returns? Very little evidence exists for the spatial dimension of the resource allocation problem, let alone both dimensions combined. \newline 

To provide evidence addressing these challenges, the third part of this dissertation collects and analyzes very detailed geospatial data on infrastructure, economic activity, and household welfare. The first paper \citep{krantz2023mapping} builds a granular 'Africa Infrastructure Database' comprising $\sim$15.1 million individual objects (places of interest) categorized into 47 economic categories (26 simplified ones) and $\sim$4.4 million km of network infrastructure (mainly roads, waterways, power lines, and railways). It jointly analyzes this data with nonparametric and ML methods, focusing on uncovering spatial patterns in infrastructure allocation and complex relationships between infrastructure and household welfare. It then employs novel 'causal ML' methods to estimate the marginal effects of different types of infrastructure on household welfare, both overall and spatially, followed by an examination of the correlates and spatial patterns of these marginal effects. Spatial analysis is done at 9.7km resolution for $>$100,000 populated locations in Africa. Findings imply that the local benefits of additional infrastructure are highly variable and context-specific. The results broadly suggest that 'hard infrastructure,' such as paved roads, power, transport, and communications, is more beneficial in cities, whereas 'social infrastructure,' such as education, health, public services, and utilities, is more critical in rural areas. Market access and agglomeration effects are important forces governing these returns. Descriptive analysis further reveals that infrastructure in Africa is concentrated in urban areas and often inefficiently allocated. African cities exhibit marked heterogeneity in infrastructure, public services, and economic activities. \newline 

Predictive ML models suggest that roads are the overall most significant infrastructure predictor of household welfare in Africa. Many reports highlight the economic significance of roads in Africa, e.g., the African Economic Outlook 2024 \citep{african2024driving} estimates that transportation (roads) accounts for 72.9\% of the estimated infrastructure financing needs until 2030\footnote{See also blog post at https://www.afdb.org/en/news-and-events/scaling-financing-key-accelerating-africas-structural-transformation-73244}, and a World Bank report \citep{foster2010africa} notes a low paved road density of 31 paved road km per 100km$^2$ of land in Africa compared to 134km in other low-income countries and urges Sub-Saharan African countries to spend 1\% of GDP on roads. Roads are also the most complex infrastructure, connecting people and locations rather than providing a local service. ML approaches only observing local quantities of roads are thus not sufficient to analyze their marginal benefits, let alone the benefits of greater investments across multiple locations. \newline

Thus, I wrote an additional paper addressing Africa's road network and spatially optimal investments into it \citep{krantz2024optimal}. Using a routing engine to compute 144 million routes between $>$12,000 locations $\sim$50km apart enables a precise appraisal of the road network's local and global efficiency. The paper then considers 447 cities with more than 100,000 people and 53 international ports and derives a network graph from fastest car routes between them, comprising 315,000km of transport roads. On this network, it characterizes both market access and welfare-maximizing investments, including cost-benefit analysis for individual links and larger investment packages. Importantly, it also examines the effects of cross-border frictions on market access and optimal road investments and takes into account trade through ports. Findings imply that cross-border frictions and trade elasticities significantly shape optimal road investments. Reducing frictions yields the greatest benefits, followed by road upgrades and new construction. Sequencing matters, as reduced frictions generally increase investment returns. Returns to upgrading key roads/links are large, even under frictions. Due to the combination of multiple kinds of data into a unified spatial framework, including novel strategies to develop accurate network representations, cost-benefit analysis, analysis of border frictions, and the use of quantitative spatial models with endogenous infrastructure, this paper is the largest and most sophisticated work included in this dissertation. Notably, it is able to connect local road investments with continental economic outcomes and computes economically optimal spatial road investment allocations at continental scale. \newline

In summary, this dissertation includes four and a half articles that expound, in great detail, on critical elements of Africa's economic transformation, supported by ambitious data collection and empirical rigour. These articles individually contribute to different economic literatures on macroeconomic moderation, global and regional value chains, economic returns to infrastructure investments, and spatially optimal infrastructure investments, as elucidated further in the respective articles. A common contribution is that all articles are, in several respects, the most detailed and data-intensive examinations of the African context within their respective literatures. They also include several methodological advances; in particular, the papers on infrastructure develop and combine methodologies to jointly examine very rich and heterogeneous geospatial infrastructure data in relation to local and continental economic outcomes. My hope is that these works of economic literature not only contribute to a deeper academic understanding of the continent but also inspire concrete policies and, in the case of the infrastructure work, technological solutions and approaches to utilize granular geospatial data for economic infrastructure policymaking. 

\newpage

\bibliographystyle{apacite}
\bibliography{africas_economic_transforamtion_references} % This links to a file bibliography.bib with the citations

\end{document}